\documentclass[a4paper,12pt,openany]{article}
\usepackage{amsmath}     % Advanced math features
\usepackage{amsfonts}    % Math fonts
\usepackage{amssymb}     % Math symbols
\usepackage{graphicx}    % Including graphics
\usepackage{geometry}    % Page layout adjustments
\usepackage{fancyhdr}    % Custom headers and footers
\usepackage{titlesec}    % Title formatting
\usepackage{multicol}    % Multi-column formatting
\usepackage{bm}

% Page layout settings
\geometry{
  left=1in,
  right=1in,
  top=1in,
  bottom=1in,
}
% Single variable partial derivative: ∂f/∂x
\newcommand{\pd}[2]{\dfrac{\partial #1}{\partial #2}}

% Second-order mixed partial derivative: ∂²f/∂x∂y
\newcommand{\pdm}[3]{\dfrac{\partial^2 #1}{\partial #2 \partial #3}}

% Higher-order partial derivative with a specified order: ∂ⁿf/∂xⁿ
\newcommand{\pdn}[3]{\dfrac{\partial^{#1} #2}{\partial #3^{#1}}}

\newcommand{\R}{\mathbb{R}}
\newcommand{\tsp}[3]{
    \vec{#1}\vec{#2}\vec{#3}
}

\newcommand{\dotp}[2]{
    \vec{#1} \cdot \vec{#2}
}

\newcommand{\crossp}[2]{
    \vec{#1} \times \vec{#2}
}

\newcommand{\vcomponents}[4]{
    #1 = \left\langle #2, #3, #4 \right\rangle
}
\newcommand{\grad}[1]{
    \nabla #1
}
\newcommand{\uniti}{
    \hat{\iota}
}
\newcommand{\unitj}{
    \hat{\jmath}
}
\newcommand{\unitk}{
    \hat{k}
}
\newcommand{\dirder}[2]{
    #1\prime_\vec{#2}
}
\newcommand{\abs}[1]{
    \left\vert#1\right\vert
}
\newcommand{\magn}[1]{
    \left\Vert#1\right\Vert
}
\newcommand{\parenth}[1]{
    \left(#1\right)
}

% Custom header and footer
\setlength{\headheight}{14.5pt}
\addtolength{\topmargin}{-2.5pt}
\pagestyle{fancy}
\fancyhf{}
\fancyhead[L]{Silas Kinnear} % Replace with your name
\fancyhead[C]{Homework \#13}
\fancyhead[R]{\thepage}


\begin{document}

\fancypagestyle{plain}{
  \fancyhf{}
  \fancyhead[L]{\leftmark}
  \fancyhead[R]{Kinnear \thepage}
}
\parskip=.5cm
\parindent=0cm

I started doing my notes in \LaTeX \, because this is how I usually take my notes and thought it would make more sense.
\section*{7.30}
\begin{align*}
    (a) \quad & \int_{-\infty}^{\infty} 4t^2 \delta(t-1) \, dt &&= 4t^2 \, \Big|_{t=1} &&= 4 \\
    (b) \quad & \int_{-\infty}^{\infty} 4t^2 \cos(2\pi t) \delta(t-0.5) \, dt &&= \cos(2\pi t) 4t^2 \, \Big|_{t=0.5} &&= -1
\end{align*}

\section*{7.42}
($a$)
\begin{align*}
    v_o(0) &= 0 \\
    v_o(\infty) &= \dfrac{4}{6} \,\cdot \, 12 = 8 \\
    v_o(t) &= v_o(\infty) + (v_o(0) - v_o(\infty))e^{-t/\tau} \\
    \tau &= R_{eq} C_{eq} \\
    R_{eq} &= 2 \parallel 4 = \dfrac{2 \cdot 4}{2 + 4} = \dfrac{8}{6} \\
    \tau &= 4 \\
    v_o(t) &= 8 - 8e^{-t/4} \text{ V}
\end{align*}
($b$)
\begin{align*}
    v_o(\infty) &= 0 \\
    v_o(t) &= v_o(0)e^{-t/\tau} \\
    v_o(0) &= \dfrac{4}{6} \, \cdot \, 12 = 8 \\
    \tau &= 4 \cdot 3 = 12 \\
    v_o(t) &= 8e^{-t/12} \text{ V}
\end{align*}

\section*{7.48}
\begin{align*}
    v(0) &= 10 \text{ for } t < 0 \\
    u(-t) &= 0 \text{ for } t < 0 \\
    v(\infty) &= 0 \text{ for } t > 0 \\
    u(-t) &= 0 \text{ for } t > 0 \\
    R_{th} &= 10 + 20 \\
    \tau &= R_{th} C = 30 \cdot 0.1 = 3 \\
    --- \quad v(t) &= 10e^{-t/3} \text{ V}\quad --- \\
    i(t) &= C \dfrac{dv}{dt} = 0.1 \dfrac{d}{dt} \parenth{10e^{-t/3}} \\
    --- \quad i(t) &= -\dfrac{1}{3}e^{-t/3} \text{ A} \quad ---
\end{align*}

\section*{7.60}
\begin{align*}
    \text{For } t < 0, \quad u(t) &= 0 \\
    i(0) &= 0\\
    \text{For } t > 0, R_{eq} &= 5\parallel 20 = 4 \Omega \\
    \tau &= \dfrac{8}{4} = 2 \\
    i(\infty) &= 8 \text{ A} \\
    i(t) &= 8(1 - e^{-t/2}) \text{ A} \\
    v(t) &= L \dfrac{di}{dt} = 8 \cdot -4 \cdot \dfrac{-1}{2} \cdot e^{-t/2} \qquad \qquad \\
    &= 16e^{-t/2} \text{ V}
\end{align*}

\section*{7.62}
\begin{align*}
    \tau &= \dfrac{2}{3 \parallel 6} = 1 \\
    \text{For $t \in (0, 1)$, $u(t-1) = 0$ : }& \\
    i(0) &= 0 \\
    i(\infty) &= \dfrac{1}{6} \\
    i(t) &= \dfrac{1}{6}(1 - e^{-t}) \text{ A} \\
    \text{For $t \in (1, \infty)$,} \\
    i(1) &= \dfrac{1}{6}(1 - e^{-1}) \\
    &\approx 0.105 \\
    i(\infty) &= \dfrac{1 + 2}{6} = \dfrac{1}{2}\\
    i(t) &= \dfrac{1}{2} + (0.105 - \dfrac{1}{2})e^{-t} \text{ A} \\
    \text{For $t \in (0, \infty)$,}& \\
    i(t) &=
    \begin{cases}
        i(t) = \dfrac{1}{6}(1 - e^{-t}) \text{ A} \qquad \text{for } t \in (0, 1) \\
        \\
        i(t) = \dfrac{1}{2} + (0.105 - \dfrac{1}{2})e^{-(t-1)} \text{ A} \quad \text{for } t \in (1, \infty)
    \end{cases}
\end{align*}



\end{document}
