\chapter*{Seven Handouts}
\section{Handout 1: Domain of Elementary Functions}

\begin{alignat*}{2}
    f &= \dfrac{u}{v} &\quad &\Rightarrow v \neq 0 \\
    f &= \sqrt[2k]{u} &\quad &\Rightarrow u \geq 0 \quad (k \in \mathbb{N}) \\
    f &= \log_{v}{u} &\quad &\Rightarrow u, v > 0, v \neq 1 \\
    f &= \tan(u) &\quad &\Rightarrow u \neq \dfrac{\pi}{2} + k\pi \quad (k \in \mathbb{Z}) \\
    f &= \cot(u) &\quad &\Rightarrow u \neq k\pi \quad (k \in \mathbb{Z}) \\
    f &= \arcsin(u) &\quad &\Rightarrow -1 \leq u \leq 1 \\
    f &= \arccos(u) &\quad &\Rightarrow -1 \leq u \leq 1 \\
    f &= u^v &\quad &\Rightarrow u > 0 \text{ or } v \in \mathbb{N} 
\end{alignat*}
Important fact: All elementary functions are continuous on their domain except at the isolated points.
$v \neq $ constant or $v = $ irrational constant.

\section{Handout 2: Fundamental Formulas for Integration}

\begin{alignat*}{8}
    &\int u^n u' \, dx  &&= \int u^n \, du 
    &&= \dfrac{u^{n+1}}{n+1} + C &\text{for } n \neq -1 \\
    &\int \dfrac{u'}{u} \, dx &&= \int \dfrac{du}{u} 
    &&= \ln|u| + C & \\
    &\int a^u u' \, dx &&= \int a^u \, du 
    &&= \dfrac{a^u}{\ln a} + C &\text{for } a > 0, a \neq 1 \\
    &\int \sin(u) \, dx && &&= -\cos(u) + C & \\
    &\int \cos(u) \, dx && &&= \sin(u) + C & \\
    &\int \dfrac{1}{\cos^2(u)} \, dx 
    &&= \int \sec^2(u) \, du &&= \tan(u) + C & \\
    &\int \dfrac{u'dx}{u^2+a^2} && = \dfrac{1}{a} \arctan\left(\dfrac{u}{a}\right) + C_1 
    &&= -\dfrac{1}{a}arccot\dfrac{u}{a} + C_2 & \\
    &\int \dfrac{u'dx}{a^2-u^2} && &&= \dfrac{1}{2a}\ln\left|\dfrac{a+u}{a-u}\right| + C_1\\
    &\int \dfrac{u'dx}{\sqrt{a^2-u^2}} &&= \arcsin\left(\dfrac{u}{a}\right) + C_1
    &&= -\arccos\left(\dfrac{u}{a}\right) + C_2 & \\
    &\int \dfrac{u'dx}{u\sqrt{u^2\pm a^2}} &&= \dfrac{1}{a}\ln\left|u+\sqrt{u^2\pm a^2}\right| + C_1
    &&= -\dfrac{1}{a}\ln\left|u+\sqrt{u^2\pm a^2}\right| + C_2 & \\
    &\int \dfrac{u'dx}{u\sqrt{a^2-u^2}} &&= \dfrac{1}{a}\arcsin\left(\dfrac{u}{a}\right) + C_1 &&= -\dfrac{1}{a}\arccos\left(\dfrac{u}{a}\right) + C_2
\end{alignat*}

\section{Handout 3: Formulas, definitions, and equations}
\subsection{Basic Trigonometric Equations}
\begin{alignat*}{8}
    &\sin(x) = a 
    &&\text{ for } a \in [-1, 1]
    &&\Leftrightarrow x =(-1)^n\arcsin(a) + 2n\pi\quad 
    &&n \in \mathbb{Z} \\
    &\sin(x) = 0 
    &&
    &&\Leftrightarrow x = n\pi 
    &&n \in \mathbb{Z} \\
    &\sin(x) = \pm 1 &&
    &&\Leftrightarrow x = \pm\dfrac{\pi}{2} + 2n\pi
    &&n \in \mathbb{Z} \\
    &\cos(x) = a 
    &&\text{ for } a \in [-1, 1]
    &&\Leftrightarrow x =(-1)^n\arccos(a) + 2n\pi 
    &&n \in \mathbb{Z} \\
    &\cos(x) = 0
    &&
    &&\Leftrightarrow x = \dfrac{\pi}{2} + n\pi
    &&n \in \mathbb{Z} \\
    &\cos(x) = \pm 1
    &&
    &&\Leftrightarrow x = 2n\pi
    &&n \in \mathbb{Z} \\
    &\tan(x) = a
    &&\text{ for } a \in \mathbb{R}
    &&\Leftrightarrow x = \arctan(a) + n\pi
    &&n \in \mathbb{Z} \\
    &\cot(x) = a
    &&\text{ for } a \in \mathbb{R}
    &&\Leftrightarrow x = \text{arccot}(a) + n\pi
    &&n \in \mathbb{Z}
\end{alignat*}
    
\subsection{Inverse Trigonometric Functions}

\begin{alignat*}{8}
    &\arcsin(x) = a
    &&\text{ for } a \in [-\dfrac{\pi}{2}, \dfrac{\pi}{2}]
    &&\Leftrightarrow x = \sin(a)
    &&\text{ for } x \in [-1, 1] \\
    &\arccos(x) = a
    &&\text{ for } a \in [0, \pi]
    &&\Leftrightarrow x = \cos(a)
    &&\text{ for } x \in [-1, 1] \\
    &\arctan(x) = a
    &&\text{ for } a \in (-\dfrac{\pi}{2}, \dfrac{\pi}{2})
    &&\Leftrightarrow x = \tan(a)
    &&\text{ for } x \in \mathbb{R} \\
    &\text{arccot}(x) = a
    &&\text{ for } a \in (0, \pi)
    &&\Leftrightarrow x = \cot(a)
    &&\text{ for } x \in \mathbb{R}
\end{alignat*}

\subsection{Other Formulas}
\[
    \sqrt{a^2} = |a|, \qquad e^{a\ln(b)} = b^a, \qquad \log_a(b) = \dfrac{\ln(b)}{\ln(a)}
\]
\pagebreak
\section{Handout 4: Derivatives}

\[
    f'(x) = \lim_{z \to x} \dfrac{f(z) - f(x)}{z-x} = \lim_{h \to 0} \dfrac{f(x+h) - f(x)}{h}
\]

\begin{multicols}{2}
    \begin{align*}
        c' &= 0 \\
        x' &= 1 \\
        (u^n)' &= nu^{n-1}u' \\
        (\sqrt{u})' &= \dfrac{u'}{2\sqrt{u}} \\
        (\sin{u})' &= \cos{u}u' \\
        (\cos{u})' &= -\sin{u}u' \\
        (\tan{u})' &= \sec^2{u}u' \\
        (\cot{u})' &= -\csc^2{u}u' \\
        (\sec{u})' &= \sec{u}\tan{u}u' \\
        (\csc{u})' &= -\csc{u}\cot{u}u' \\
        (u\pm v)' &= u' \pm v' \\
        (uv)' &= u'v + uv' \\  
        (cu)' &= cu' \\
        \left(\dfrac{u}{v}\right)' &= \dfrac{u'v - uv'}{v^2} \\
        \left(\dfrac{c}{u^n}\right)' &= -\dfrac{ncu'}{u^{n+1}} \\
    \end{align*}
    \vfill\null
    \columnbreak

    \begin{align*}
        (\dfrac{u}{c})' &= \dfrac{u'}{c} \\
        (\dfrac{c}{u})' &= -\dfrac{cu'}{u^2} \\
        (e^u)' &= e^u u' \\
        (a^u)' &= a^u \ln(a)u' \\
        (\ln{u}) &= \dfrac{u'}{u} \\
        (\log_a{u})' &= \dfrac{u'}{u\ln(a)} \\
        (\arcsin{u})' &= \dfrac{u'}{\sqrt{1-u^2}} \\
        (\arccos{u})' &= -\dfrac{u'}{\sqrt{1-u^2}} \\
        (\arctan{u})' &= \dfrac{u'}{1+u^2} \\
        (\text{arccot}{u})' &= -\dfrac{u'}{1+u^2} \\
    \end{align*}
\end{multicols}
\pagebreak
\section{Handout 5: Trigonometry}

\subsection{Trigonometric Identities}

\begin{align*}
    \sin\alpha\cos\beta &= \dfrac{1}{2}\sin{(\alpha-\beta)} + \dfrac{1}{2}\sin{(\alpha+\beta)} \\
    \cos\alpha\cos\beta &= \dfrac{1}{2}\cos{(\alpha-\beta)} + \dfrac{1}{2}\cos{(\alpha+\beta)} \\
    \sin\alpha\sin\beta &= -\dfrac{1}{2}\cos{(\alpha+\beta)} + \dfrac{1}{2}\cos{(\alpha-\beta)} \\
    \vspace{.25in}\\
    \cos^2\alpha &= \dfrac{1+\cos{2\alpha}}{2} \\
    \sin^2\alpha &= \dfrac{1-\cos{2\alpha}}{2} \\
    \vspace{.25in}\\
    \sin{(\alpha \pm \beta)} &= \sin\alpha\cos\beta \mp \cos\alpha\sin\beta \\
    \cos{(\alpha \pm \beta)} &= \cos\alpha\cos\beta \mp \sin\alpha\sin\beta \\
    \tan{(\alpha \pm \beta)} &= \dfrac{\tan\alpha \pm \tan\beta}{1 \mp \tan\alpha\tan\beta}\\
    \vspace{.25in}\\
    \sin{(2\alpha)} &= 2\sin\alpha\cos\alpha \\
    \cos{(2\alpha)} &= \cos^2\alpha - \sin^2\alpha \\
    \tan{(2\alpha)} &= \dfrac{2\tan\alpha}{1-\tan^2\alpha}\\
    \vspace{.25in}\\
    \sin(\alpha) \pm \sin(\beta) &= 2\sin\left(\dfrac{\alpha \pm \beta}{2}\right)\cos\left(\dfrac{\alpha \mp \beta}{2}\right) \\
    \cos(\alpha) \pm \cos(\beta) &= 2\cos\left(\dfrac{\alpha + \beta}{2}\right)\cos\left(\dfrac{\alpha \mp \beta}{2}\right) \\
\end{align*}
How to exit trigonometry:
\begin{alignat*}{8}
    \tan{\dfrac{x}{2}} = y &\Rightarrow \sin x &&= \dfrac{2y}{1+y^2}, &\cos x &= \dfrac{1-y^2}{1+y^2}, & dx &= \dfrac{2dy}{1+y^2} &&\text{ for } x \neq (2k+1)\pi\\
    \tan x = y &\Rightarrow \sin^2 x &&= \dfrac{y}{1+y^2},\,\, &\cos^2 x &= \dfrac{1}{1+y^2},\,\, & dx &= \dfrac{dy}{1+y^2} &&\text{ for } x \neq \dfrac{\pi}{2} + k\pi\\
\end{alignat*}
\pagebreak
\subsection{Trigonometric Chart}
\begingroup
\linespread{2}\selectfont  % Adjust the line spacing to 1.5 times the normal

\begin{table*}[h!]
\centering
\begin{tabular}{|c|c|c|c|c|}
\hline
\textbf{Angle} & \textbf{Sine} & \textbf{Cosine} & \textbf{Tangent} & \textbf{Cotangent} \\
\hline
$-\dfrac{\pi}{2} - 0$ & $-1$ & $-0$ & $+\infty$ & $+0$ \\
$-\dfrac{\pi}{2} + 0$ & $-1$ & $+0$ & $-\infty$ & $-0$ \\
$-0$ & $-0$ & $1$ & $-0$ & $-\infty$ \\
$+0$ & $+0$ & $1$ & $+0$ & $+\infty$ \\
$\dfrac{\pi}{6}$ & $\dfrac{1}{2}$ & $\dfrac{\sqrt{3}}{2}$ & $\dfrac{1}{\sqrt{3}}$ & $\sqrt{3}$ \\
$\dfrac{\pi}{4}$ & $\dfrac{\sqrt{2}}{2}$ & $\dfrac{\sqrt{2}}{2}$ & $1$ & $1$ \\
$\dfrac{\pi}{3}$ & $\dfrac{\sqrt{3}}{2}$ & $\dfrac{1}{2}$ & $\sqrt{3}$ & $\dfrac{1}{\sqrt{3}}$ \\
$\dfrac{\pi}{2} - 0$ & $1$ & $+0$ & $+\infty$ & $+0$ \\
$\dfrac{\pi}{2} + 0$ & $1$ & $-0$ & $-\infty$ & $-0$ \\
$\pi - 0$ & $+0$ & $-1$ & $-0$ & $-\infty$ \\
$\pi + 0$ & $-0$ & $-1$ & $+0$ & $+\infty$ \\
\hline
\end{tabular}
\end{table*}

\pagebreak
\section{Handout 6: Power Series for Elementary Functions}

\begin{alignat*}{4}
    \dfrac{1}{1-x} &= \sum_{n=0}^{\infty} x^n, \quad &&x\in (-1, 1) \\
    e^x &= \sum_{n=0}^\infty\dfrac{x^n}{n!}, &&x \in(-\infty, +\infty) \\
    \sin x &= \sum_{n=1}^\infty (-1)^{n-1} \dfrac{x^{2n-1}}{(2n-1)!}, \quad&&x \in (-\infty, +\infty) \\
    \cos x &= \sum_{n=0}^\infty (-1)^{n} \dfrac{x^{2n}}{(2n)!}, \quad&&x \in (-\infty, +\infty) \\
    \ln{(1+x)} &= \sum_{n=1}^\infty (-1)^{n-1} \dfrac{x^n}{n}, \quad&&x \in \left(-1, 1\right] \\
    \arctan x &= \sum_{n=0}^\infty (-1)^n \dfrac{x^{2n+1}}{2n+1}, \quad&&x \in [-1, 1] \\
    (1+x)^\alpha &= \sum_{n=0}^\infty \binom{\alpha}{n} x^n, \quad&&x \in (-1, 1) \\
    \binom{\alpha}{n} &= \dfrac{\alpha(\alpha-1)\cdots(\alpha-(n-1))}{n!} \\
    \arcsin x &= \sum_{n=0}^\infty \dfrac{(2n)!}{4^n(n!)^2(2n+1)}x^{2n+1}, \quad&&x \in [-1, 1]
\end{alignat*}

\pagebreak
\section*{Handout 7.1: Application of Various Types of Integrals in Mechanics}
for mass, center of mass, centroid, moments of statics, moments of inertia, length, distance, area, volume, surface area, etc.

of an object $O$ the following types:
\begin{itemize}
    \item Arbitrary 3D Solid;
    \item Lamina (2D in 2D);
    \item Rod (1D in 1D);
    \item Wire (1D in 3D);
    \item Shell (2D in 3D);
    \item Points (0D in 3D);
\end{itemize}
with respect to $Q$, as $Q$ could be a point, a straight line, or a plane.

All applications for the general case of inhomogeneous objects and the special case of homogeneous objects are given by only one formula.
\pagebreak
\section*{Handout 7.1}
\begin{equation*}
    mom_Q^{(n)}(O) = \int_O D^n \delta \, dE
\end{equation*}

where $O$ is the object; $Q$ is a point, straight line, or plane; $\delta$ is the density; $dE$ is a portion of $O$; $D$ is the distance from an arbritary point $P$ ($P \in O$) to $Q$.
\begin{itemize}
    \item $n = 0$: mass, area, length, volume, etc.
    \item $n = 1$: moment of statics
    \item $n = 2$: moment of inertia
\end{itemize}
\endgroup