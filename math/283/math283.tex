\documentclass[a4paper,12pt,openany]{book}
\usepackage{amsmath}    % Advanced math features
\usepackage{amsfonts}   % Math fonts
\usepackage{amssymb}    % Math symbols
\usepackage{graphicx}   % Including graphics
\usepackage{geometry}   % Page layout adjustments
\usepackage{fancyhdr}   % Custom headers and footers
\usepackage{titlesec}   % Title formatting
\usepackage{multicol}   % Multi-column formatting

% Page layout settings
\geometry{
  left=1in,
  right=1in,
  top=1in,
  bottom=1in,
}
% Single variable partial derivative: ∂f/∂x
\newcommand{\pd}[2]{\frac{\partial #1}{\partial #2}}

% Second-order mixed partial derivative: ∂²f/∂x∂y
\newcommand{\pdm}[3]{\frac{\partial^2 #1}{\partial #2 \partial #3}}

% Higher-order partial derivative with a specified order: ∂ⁿf/∂xⁿ
\newcommand{\pdn}[3]{\frac{\partial^{#1} #2}{\partial #3^{#1}}}

\newcommand{\R}{\mathbb{R}}
\newcommand{\tsp}[3]{
    \vec{#1}\vec{#2}\vec{#3}
}

\newcommand{\dotp}[2]{
    \vec{#1} \cdot \vec{#2}
}

\newcommand{\crossp}[2]{
    \vec{#1} \times \vec{#2}
}

\newcommand{\vcomponents}[4]{
    #1 = \langle #2, #3, #4 \rangle
}
\newcommand{\grad}[1]{
    \nabla #1
}
\newcommand{\uniti}{
    \hat{\iota}
}
\newcommand{\unitj}{
    \hat{\jmath}
}
\newcommand{\unitk}{
    \hat{k}
}
\newcommand{\dirder}[2]{
    #1\prime_\vec{#2}
}
% Custom header and footer
\pagestyle{fancy}
\fancyhf{}
\fancyhead[L]{Silas Kinnear} % Replace with your name
\fancyhead[C]{MATH 283}
\fancyhead[R]{\thepage}

% Title format
\titleformat{\chapter}[block]{\LARGE\bfseries}{\thechapter.}{1em}{}
\titleformat{\section}[block]{\Large\bfseries}{\thesection}{1em}{}
\titleformat{\subsection}[block]{\large\bfseries}{\thesubsection}{1em}{}

\begin{document}

\title{Calculus III - MATH 283}
\author{Silas Kinnear}
\date{Fall 2024}
\maketitle

\tableofcontents
\fancypagestyle{plain}{
  \fancyhf{}
  \fancyhead[L]{\leftmark}
  \fancyhead[R]{Kinnear \thepage}
}
\parskip=.5cm
\parindent=0cm
%-------------------------------
% Chapter 1: Elementary Functions
%-------------------------------
\chapter{Elementary Functions}

\section{Classifications}
All functions belong to specific classifications. Algebraic functions include rational functions, and rational functions consist of polynomial functions.

\begin{itemize}
    \item \textbf{Polynomial Functions}
    \begin{itemize}
        \item Addition
        \item Subtraction
        \item Multiplication
    \end{itemize}
    \item \textbf{Rational Functions}
    \begin{itemize}
        \item Division
    \end{itemize}
    \item \textbf{Algebraic Functions}
    \begin{itemize}
        \item Rational Powers
    \end{itemize}
\end{itemize}
All of these operations combined a finite number of times in one formula are known as elementary functions.

\section{Limits and Continuity}

\begin{align*}
    \lim_{x \to 0} \frac{1}{x} &= \infty \quad \text{(DNE)} \\
    \lim_{x \to 0^+} \ln(x) &= -\infty \quad \text{(DNE)} \\
    \lim_{x \to 0^-} \ln(x) &= \text{DNE}
\end{align*}

%-------------------------------
% Chapter 2: Dimensional Analysis
%-------------------------------
\chapter{Dimensional Analysis}

\section{Introduction}
Dimensional analysis is a fundamental tool in understanding the relationships between different physical quantities.

\section{Classifications}

\begin{align*}
    \text{Circle:} \quad & (x-h)^2 + (y-k)^2 = r^2 \\
    \text{Sphere:} \quad & (x-h)^2 + (y-k)^2 + (z-l)^2 = r^2 \\
    \text{Disk:} \quad & (x-h)^2 + (y-k)^2 \leq r^2
\end{align*}
Here we ask: How many dimensions are needed to visualize these objects? The answer depends on the number of variables in the equation.

\begin{itemize}
    \item{\textbf{Plane Lines:} 2 variables, 1 equation}
    \item{\textbf{Space Lines:} 3 variables, 2 equations}
\end{itemize}
Examples of planes:
\[
\text{xy-plane : } z = 0
\]
\[
\text{xz-plane : } y = 0
\]
\[
\text{yz-plane : } x = 0
\]
Examples of lines:
\[
\text{x-axis : }
\begin{cases}
    z = 0\\
    y = 0
\end{cases}
\]
\[
\text{y-axis : }
\begin{cases}
    x = 0\\
    z = 0
\end{cases}
\]
\[
\text{z-axis : }
\begin{cases}
    x = 0\\
    y = 0
\end{cases}
\]
\pagebreak
\section{Geometric Equations}

\subsection{Distance Between Two Points}

The distance \(d_{PQ}\) between two points \( P(x_1, y_1, z_1) \) and \( Q(x_2, y_2, z_2) \) in three-dimensional space is given by:
\begin{equation}
    d_{PQ} = \sqrt{(x_2 - x_1)^2 + (y_2 - y_1)^2 + (z_2 - z_1)^2}
\end{equation}

\subsection{Midpoint Formula}

The midpoint \(M_{PQ}\) between two points \( P(x_1, y_1, z_1) \) and \( Q(x_2, y_2, z_2) \) is calculated as:
\begin{equation}
    M_{PQ} = \left(\frac{x_1 + x_2}{2}, \frac{y_1 + y_2}{2}, \frac{z_1 + z_2}{2}\right)
\end{equation}

\subsection{Equation of a Sphere}

A sphere centered at \(P(p, q, s)\) with radius \( r \) has the equation:
\begin{equation}
    (x - p)^2 + (y - q)^2 + (z - s)^2 = r^2
\end{equation}

\subsection{Equation of an Ellipsoid}

An ellipsoid centered at \(P(h, k, l)\) is given by:

\begin{equation}
    \frac{(x-h)^2}{a^2} +  
    \frac{(y-k)^2}{b^2} + 
    \frac{(z-l)^2}{c^2} = 1
\end{equation}

\subsection{Equations of a Line}
\(\vec{a_l} = \langle a, b, c \rangle\) is the vector collinear to the line, and \(P(x_0, y_0, z_0)\) is a point on the line.
\subsubsection{Parametric Form}
\begin{equation}
    l = 
    \begin{cases}
        x = x_0 + at\\
        y = y_0 + bt\\
        z = z_0 + ct
    \end{cases}
\end{equation}

\subsubsection{Normal Form}
\begin{equation}
    \frac{x-x_0}{a} = \frac{y-y_0}{b} = \frac{z-z_0}{c}
\end{equation}

\subsection{Equations of a Plane}
\(\vec{a_l} = \langle a, b, c \rangle\) is the vector normal to the plane, and \(P(x_0, y_0, z_0)\) is a point on the plane.
\subsubsection{Point-Normal/Scalar Form}
\begin{equation}
    a(x-x_0) + b(y-y_0) + c(z-z_0) = 0
\end{equation}
\section{Graphing Concepts}

Graphing helps visualize functions and equations by showing the set of all points that satisfy them.

\subsection{Functions of a Single Variable}

For a function of a single variable \( f(x) \):
\begin{itemize}
    \item \textbf{Domain}: A subset or all of the real number line, often denoted as \(\mathbb{R}\) or a specific interval such as \((- \infty, \infty)\).
    \item \textbf{Range}: The set of possible values of \( f(x) \); for instance, \([0, \infty)\).
    \item \textbf{Graph}: A curve on the Cartesian plane, representing ordered pairs \((x, f(x))\).
\end{itemize}

\subsection{Functions of Multiple Variables}

For functions of two or more variables, the domain and range extend to higher dimensions:
\begin{itemize}
    \item \textbf{Domain}: The set of all points in \(\mathbb{R}^n\) (e.g., \(\mathbb{R}^2\) for a function of two variables or \(\mathbb{R}^3\) for three variables) where the function is defined.\\
        - \textbf{Entire Plane}: If the function \( f(x, y) \) is defined for all \((x, y) \in \mathbb{R}^2\).\\
        - \textbf{Portion of the Plane}: A subset of \(\mathbb{R}^2\), specified by conditions like \(x > 0\) or \(y \geq 1\).
    \item \textbf{Range}: The set of output values of the function. For many functions of multiple variables, this is a subset of \(\mathbb{R}\).
    \item \textbf{Graph}: For a function \( f(x, y) \) in two variables, the graph is a surface in three-dimensional space. For functions of three variables, the graph would exist in four-dimensional space and cannot be directly visualized.
\end{itemize}

\subsection*{Examples of Functions of Multiple Variables}

\textbf{1: Linear Function of Two Variables}
    \[
    f(x, y) = 3x + 5y - 7
    \]
    \begin{itemize}
        \item \textbf{Domain}: \(\mathbb{R}^2\) (all real pairs \((x, y)\))
        \item \textbf{Range}: \(\mathbb{R}\) (all real values)
        \item \textbf{Graph}: A plane in three-dimensional space.
    \end{itemize}

\textbf{2: Linear Function of Three Variables}
    \[
    f(x, y, z) = 3x - 5y + z - 11
    \]
    \begin{itemize}
        \item \textbf{Domain}: \(\mathbb{R}^3\) (all real triples \((x, y, z)\))
        \item \textbf{Range}: \(\mathbb{R}\)
        \item \textbf{Graph}: Exists in four-dimensional space; it cannot be visualized in three dimensions.
    \end{itemize}
\pagebreak
\section{Graphing Dimensions Summary}

Graphing dimensions change based on the variables involved:
\begin{itemize}
    \item \textbf{1D}: Interval or union of intervals
    \item \textbf{2D}: Picture
    \item \textbf{3D}: Description
    \item \textbf{4D and Higher}: Equation
\end{itemize}

%-------------------------------
% Chapter 3: Functions of Several Variables
%-------------------------------
\chapter{Functions of Several Variables}
Functions of several variables extend the concept of functions to higher dimensions, allowing for more complex mappings and dependencies.

% Example content for introduction, definitions, and properties of functions of several variables.

%-------------------------------
% Chapter 4: Partial Derivatives
%-------------------------------
\chapter{Partial Derivatives}
Partial derivatives are used to study functions with multiple variables by differentiating with respect to one variable while keeping the others constant.

% Example content for partial derivative rules, examples, and applications.

%-------------------------------
% Chapter 5: Multiple Integrals
%-------------------------------
\chapter{Multiple Integrals}
Multiple integrals extend single-variable integration to functions of several variables, useful in areas such as volume and area calculations.

% Example content for double and triple integrals, as well as change of variables.

%-------------------------------
% Chapter 6: Vector Calculus
%-------------------------------
\chapter{Vector Calculus}
Vector calculus explores vector fields and operations such as the gradient, divergence, and curl, which are foundational in physics, engineering, and mathematics.

\section{Introduction to Vectors}
Vectors represent quantities with both magnitude and direction. Key operations include scalar multiplication, addition, and vector products like the dot and cross products.

\subsection{Scalar Multiples}
A scalar multiple of a vector \(\vec{u}\) scales its magnitude without changing its direction.

\section{Dot Product}
The dot product is an operation between two vectors \(\vec{u}\) and \(\vec{v}\) that produces a scalar, and is calculated as:
\[
    \vec{u} \cdot \vec{v} = u_x v_x + u_y v_y + \cdots + u_n v_n = ||\vec{u}|| \, ||\vec{v}|| \cos{\theta}
\]
where \(\theta\) is the angle between \(\vec{u}\) and \(\vec{v}\). The dot product is \textbf{commutative}:
\[
    \vec{u} \cdot \vec{v} = \vec{v} \cdot \vec{u}
\] 

\subsection{Applications of the Dot Product}
\begin{enumerate}
    \item \textbf{Magnitude of a Vector}:
    \[
        \vec{a} \cdot \vec{a} = ||\vec{a}||^2
    \]
    \[
        ||\vec{a}|| = \sqrt{\vec{a} \cdot \vec{a}}
    \]

    \item \textbf{Determining Perpendicularity}:
    \[
        \vec{a} \cdot \vec{b} = 0 \iff \vec{a} \perp \vec{b}
    \]
    For example, if \(\vec{a}_{L_1} \cdot \vec{a}_{L_2} = 0\), then lines \(L_1\) and \(L_2\) are perpendicular.
\end{enumerate}

\section{Cross Product}
The cross product is an operation on two 3D vectors that yields a vector perpendicular to both:
\[
    \vec{u} \times \vec{v} = 
    \begin{vmatrix}
        \uniti & \unitj & \unitk \\
        u_x & u_y & u_z \\
        v_x & v_y & v_z \\
    \end{vmatrix} 
    = \langle u_y v_z - u_z v_y, \; u_z v_x - u_x v_z, \; u_x v_y - u_y v_x \rangle
\]
This resultant vector \(\vec{w} = \vec{u} \times \vec{v}\) is perpendicular to both \(\vec{u}\) and \(\vec{v}\).
\subsection{Shortcut for Cross Product}

To find the cross product \(\vec{u} \times \vec{v}\), arrange the components as follows:
\[
\begin{array}{|cccccc|}
    u_x & u_y & u_z & u_x & u_y & u_z \\
    v_x & v_y & v_z & v_x & v_y & v_z \\
\end{array}
\]

Then, calculate each component of the cross product by making crosses between $u_y$ and $v_z$, $u_z$ and $v_y$, $u_z$ and $v_x$, $u_x$ and $v_z$, $u_x$ and $v_y$, and $u_y$ and $v_x$. This yields the components of the cross product vector and is easy to visualize.

\section{Directional Derivatives}
\subsection{Definition}
The directional derivative of \(f\) at a point \((x_0, y_0, z_0)\) in the direction of a vector \(\vec{v}\) is defined as:
\[
    f'_{\vec{v}}(x_0, y_0, z_0) = 
    \lim_{h \to 0}
    \frac{
        f(x_0 + h \cos \alpha, y_0 + h \cos \beta, z_0 + h \cos \gamma) - f(x_0, y_0, z_0)
    }{h}
\]
where \(\cos \alpha\), \(\cos \beta\), and \(\cos \gamma\) are the \textbf{directional cosines} of \(\vec{v}\).

\subsection{Shortcut Formula for Directional Derivative}
Using the gradient, the directional derivative can be computed as:
\[
    f'_{\vec{v}}(x_0, y_0, z_0) = \vec{u} \cdot \nabla f(x_0, y_0, z_0)
\]
where:
\[
    \nabla f = \left\langle \frac{\partial f}{\partial x}, \frac{\partial f}{\partial y}, \frac{\partial f}{\partial z} \right\rangle
\]
The gradient vector, \(\nabla f\), points in the direction of greatest increase of \(f\) and has a magnitude equal to the maximum rate of increase.

\end{document}
