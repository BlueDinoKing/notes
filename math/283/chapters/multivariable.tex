\chapter{Functions of Several Variables}
Functions of several variables extend the concept of functions to higher dimensions, allowing for more complex mappings and dependencies.

\section{Continuity}

4 conditions for continuity of $f(x, y)$ at a point $(x_0, y_0)$:

\begin{itemize}
    \item $(x_0, y_0)$ is in the domain of $f(x, y)$
    \item $(x_0, y_0)$ is not an isolated point
    \item $\lim_{\substack{x \to x_0\\y \to y_0}} f(x, y)$ exists
    \item $\lim_{\substack{x \to x_0\\y \to y_0}} f(x, y) = f(x_0, y_0)$
\end{itemize}

\section{Equations}

\subsection{Equations of a Tangent Plane}
\begin{equation} \label{Tangent Plane}
    z = f(x_0, y_0) + f_x(x_0, y_0)(x - x_0) + f_y(x_0, y_0)(y - y_0)
\end{equation}
\begin{itemize}
    \item Given $P(x_0, y_0, z_0)$ and $f(x, y) = x^2 + y\sin x$
    \begin{align*}
        f(x,y) &= x^2 + y\sin x \\
        z &= x^2 + y\sin x \\
        F(x, y, z) &= z - x^2 - y\sin x
    \end{align*}
    \item $\nabla F = 
    \begin{bmatrix} 
    -2x - y\cos x \\ 
    -\sin x \\ 1 
    \end{bmatrix}$
\end{itemize}

\section{Finding the Tangent Plane}
Given a function $f(x, y)$, the equation of the tangent plane at a point $(x_0, y_0, z_0)$, where $z_0 = f(x_0, y_0)$, is given by:

\begin{equation}
    z - z_0 = f_x(x_0, y_0)(x - x_0) + f_y(x_0, y_0)(y - y_0)
\end{equation}

\noindent Here, $f_x$ and $f_y$ are the partial derivatives of $f(x, y)$ with respect to $x$ and $y$ respectively, evaluated at $(x_0, y_0)$. To find the tangent plane:

\begin{enumerate}
    \item Find $f_x(x, y)$ and $f_y(x, y)$.
    \item Evaluate $f_x(x_0, y_0)$ and $f_y(x_0, y_0)$.
    \item Substitute into the tangent plane equation.
\end{enumerate}

\section{Finding Critical Points}
To find the critical points of a function $f(x, y)$:

\begin{enumerate}
    \item Compute the first partial derivatives $f_x(x, y)$ and $f_y(x, y)$.
    \item Set the equations $f_x(x, y) = 0$ and $f_y(x, y) = 0$.
    \item Solve the system of equations to find the points $(x_c, y_c)$ where the gradients are zero.
\end{enumerate}

Once critical points $(x_c, y_c)$ are found, determine their nature (local maxima, local minima, or saddle points) using the second derivative test:

\begin{equation}
    D = f_{xx}(x_c, y_c) f_{yy}(x_c, y_c) - [f_{xy}(x_c, y_c)]^2
\end{equation}

\begin{itemize}
    \item If $D > 0$ and $f_{xx}(x_c, y_c) > 0$, $(x_c, y_c)$ is a local minimum.
    \item If $D > 0$ and $f_{xx}(x_c, y_c) < 0$, $(x_c, y_c)$ is a local maximum.
    \item If $D < 0$, $(x_c, y_c)$ is a saddle point.
    \item If $D = 0$, the test is inconclusive.
\end{itemize}