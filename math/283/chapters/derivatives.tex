\chapter{Partial Derivatives}
Partial derivatives are used to study functions with multiple variables by differentiating with respect to one variable while keeping the others constant.

\section{Differentiability}
A function of more than two variables is said to be
differentiable if the function all of its first partial
derivative and all of its second partial derivatives exist
and are continuous.

\section{Introduction}
Partial derivatives are used to study how a function changes with respect to one variable while keeping the others constant.

\subsection{Differentials}
The derivative of a function \(f(x)\) with respect to \(x\) is denoted as: $\dfrac{dy}{dx}$ or $f'(x)$ where dx is the differential of x and $dy = f'(x)dx$.
Differentials are different than derivatives, as they are the change in the function value due to a change in the variable. For example,
\[
\text{if }f(x) = x^2\text{, then } df = 2xdx\text{.}
\]
\[
d(\arctan(u)) = \dfrac{du}{1+u^2}\text{ but }\arctan(u') = \dfrac{u'}{1+u^2}
\]

\subsubsection{Error Estimation}
The error in a function \(f(x)\) can be estimated by the differential:
\begin{equation}\label{Error Estimation}
    df = f'(x)dx
\end{equation}
This is useful for approximating the error in a function given a small change in the variable.

\subsubsection{Example of Error Estimation}
Given the equation $A = lw$, $l = 5m$, $w = 3m$, $dl = 0.001m$, and $dw = 0.001m$, find the error in the area ($dA$): 
\begin{align*}
    A &= lw\\
    dA &= wdl + ldw\\
    dA &= (3)(0.001) + (5)(0.001)\\
    dA &= 0.008m^2
\end{align*}
\subsection{Derivatives}

\[
    f(x) \text{is differentiable iff } \lim_{h \to 0} \dfrac{f(x+h) - f(x)}{h} \text{ exists}
\]
\subsubsection{Linearization}
The linearization of a function \(f(x)\) at a point \(x_0\) is given by:
\begin{equation}\label{Linearization}
    L(x) = f(x_0) + f'(x_0)(x-x_0)
\end{equation}
\[
    f(x) \approx L(x) \text{ for } x \text{ near } x_0
\]

This is useful for approximating the value of a function near a point. $x_0$ should be an integer close to the point you are trying to approximate.
\subsubsection{Example of Linearization}
Given the function \(f(x) = \sqrt{x}\) and the point \(x_0 = 3.9\), we can approximate this by linearizing \(f(x)\) at \(x_0\):
\begin{align*}
    L(x) &= f(x_0) + f'(x_0)(x-x_0)\\
    L(x) &= \sqrt{4} + \dfrac{1}{2\sqrt{4}}(x-4)\\
    L(x) &= 2 + \dfrac{1}{4}(x-4)\\
    L(3.9) &= 1.975 \approx \sqrt{3.9} \approx 1.9748    
\end{align*}

\section{Partial Derivatives}
\subsection{Partial Derivatives Definition}

The partial derivative of a function \(f(x, y)\) with respect to \(x\) is denoted as:
\[
    \pd{f}{x} = \lim_{h \to 0} \dfrac{f(x+h, y) - f(x, y)}{h}
\]
This \textbf{not} a fraction. $\partial x$ does not exist, and neither does $\partial f$.

\begin{equation}\label{Partial Derivative at a Point Definition}
    \pd{f}{x}(x_0, y_0) = 
    \lim_{x \to x_0} \dfrac{f(x, y_0) - f(x_0, y_0)}{x-x_0}
\end{equation}
\begin{equation}\label{Partial Derivative delfdelx Definition}
    \pd{f}{x} = \lim_{h \to 0} \dfrac{f(x+h, y) - f(x, y)}{h}
\end{equation}
\begin{equation}\label{Partial Derivative delfdely Definition}
    \pd{f}{y} = \lim_{h \to 0} \dfrac{f(x, y+h) - f(x, y)}{h}
\end{equation}

\subsubsection{Shortcut}
The partial derivative of a function \(f(x, y, \cdots)\) with respect to \(x\) can be calculated by treating all other variables as a constant and differentiating with respect to \(x\):

\subsection{All Defintions of First and Second Partial Derivatives}
You sometimes have to use the definition of a partial derivative to find the actual value of the partial derivative. 
\begin{equation} \label{Partial Derivative f_x Definition}
    f_x = \pd{f}{x} = \lim_{h \to 0} \dfrac{f(x+h, y) - f(x, y)}{h}
\end{equation}
\begin{equation}\label{Partial Derivative f_y Definition}
    f_y = \pd{f}{y} = \lim_{h \to 0} \dfrac{f(x, y+h) - f(x, y)}{h}
\end{equation}
\begin{equation}\label{Second Partial Derivative f_xx Definition}
    f_{xy} = \pdm{f}{x}{y} = \lim_{h \to 0} \dfrac{\pd{f}{y}(x+h, y) - \pd{f}{y}(x, y)}{h}
\end{equation}
\begin{equation}\label{Second Partial Derivative f_yx Definition}
    f_{yx} = \pdm{f}{y}{x} = \lim_{h \to 0} \dfrac{\pd{f}{x}(x, y+h) - \pd{f}{x}(x, y)}{h}
\end{equation}
\begin{equation}\label{Second Partial Derivative f_xx Definition}
    f_{xx} = \pdm{f}{x}{x} = \lim_{h \to 0} \dfrac{\pd{f}{x}(x+h, y) - \pd{f}{x}(x, y)}{h}
\end{equation}
\begin{equation}\label{Second Partial Derivative f_yy Definition}
    f_{yy} = \pdm{f}{y}{y} = \lim_{h \to 0} \dfrac{\pd{f}{y}(x, y+h) - \pd{f}{y}(x, y)}{h}
\end{equation}

\subsection{Properties of Partial Derivatives}
If $f(x,y)$ is continuous at $(x_0, y_0)$, then $f_x$ and $f_y$ exist at $(x_0, y_0)$ and $f_{xy} = f_{yx}$.

\subsubsection{Total Differential}
The total differential of a function \(f(x, y)\) is given by:
\[
    df = \pd{f}{x}dx + \pd{f}{y}dy
\]
If you find df, it is very easy to find both partial derivatives via the coefficients of dx and dy.
\subsection{Higher-Order Partial Derivatives}
The second-order partial derivative of a function \(f(x, y)\) with respect to \(x\) and \(y\) is denoted as:
\[
    \pdm{f}{x}{y} = \pd{}{x}\left(\pd{f}{y}\right)
\]
Order does matter. For example, for some instances: \[\pdm{f}{x}{y} \neq \pdm{f}{y}{x}\] However, these are usually equal for continuous functions. It is read from left to right and can also be notated as $f_x$, \(f_{xy}\), \(f_{yx}\), etc.
\pagebreak
\section{Examples:}
\[
    f(x, y) = 
    \begin{cases}
        \dfrac{x^2 y}{x^2 + y^2}, & (x, y) \neq (0, 0)\\
        0, & (x, y) = (0, 0)
    \end{cases}
\]

For \( (x, y) \neq (0, 0) \), the partial derivatives are:
\[
    \pd{f}{x} = \dfrac{2xy^3}{(x^2 + y^2)^2}
\]
\[
    \pd{f}{y} = \dfrac{x^4 - x^2 y^2}{(x^2 + y^2)^2}
\]
\[
    \pdm{f}{x}{y} = \dfrac{2x^3 y^2 - 2x y^4}{(x^2 + y^2)^3}
\]
\[
    \pdm{f}{y}{x} = \dfrac{2x^3 y^2 - 2x y^4}{(x^2 + y^2)^3}
\]
\[
    \pdm{f}{x}{x} = \dfrac{2y^3(x^2 - y^2)}{(x^2 + y^2)^3}
\]
\[
    \pdm{f}{y}{y} = \dfrac{x^2(2x^2 - y^2)}{(x^2 + y^2)^3}
\]

For \( (x, y) = (0, 0) \), we use the limit definition of partial derivatives. The partial derivative of \( f \) with respect to \( x \) at \( (0, 0) \) is:
\[
    \pd{f}{x}(0, 0) = \lim_{h \to 0} \dfrac{f(h, 0) - f(0, 0)}{h} = \lim_{h \to 0} \dfrac{0 - 0}{h} = 0
\]
Similarly, the partial derivative of \( f \) with respect to \( y \) at \( (0, 0) \) is:
\[
    \pd{f}{y}(0, 0) = \lim_{k \to 0} \dfrac{f(0, k) - f(0, 0)}{k} = \lim_{k \to 0} \dfrac{0 - 0}{k} = 0
\]

Thus, we have:
\[
    \pd{f}{x}(0, 0) = 0, \quad \pd{f}{y}(0, 0) = 0
\]

To find the second partial derivatives at \( (0, 0) \), we proceed similarly:
\[
    \pdm{f}{x}{x}(0, 0) = \lim_{h \to 0} \dfrac{\pd{f}{x}(h, 0) - \pd{f}{x}(0, 0)}{h} = \lim_{h \to 0} \dfrac{0 - 0}{h} = 0
\]
\[
    \pdm{f}{y}{y}(0, 0) = \lim_{k \to 0} \dfrac{\pd{f}{y}(0, k) - \pd{f}{y}(0, 0)}{k} = \lim_{k \to 0} \dfrac{0 - 0}{k} = 0
\]
\[
    \pdm{f}{x}{y}(0, 0) = \pdm{f}{y}{x}(0, 0) = \lim_{h \to 0} \dfrac{\pd{f}{y}(h, 0) - \pd{f}{y}(0, 0)}{h} = 0
\]

Thus, all second partial derivatives at \( (0, 0) \) are:
\[
    \pdm{f}{x}{x}(0, 0) = 0, \quad \pdm{f}{y}{y}(0, 0) = 0, \quad \pdm{f}{x}{y}(0, 0) = 0, \pdm{f}{y}{x}(0, 0) = DNE
\]
\[
    \pdm{f}{y}{x}(0, 0) = 
    \lim_{x\to 0} \dfrac{\pd{f}{x}(x, 0) - \pd{f}{x}(0, 0)}{x} = \lim_{x\to 0} \dfrac{1 - 0}{x} = DNE
\]

All partial derviatives combined are:
\[
    \pd{f}{x} = 
    \begin{cases}
        \dfrac{2xy^3}{(x^2 + y^2)^2}, & (x, y) \neq (0, 0)\\
        0, & (x, y) = (0, 0)
    \end{cases}
\]
\[
    \pd{f}{y} = 
    \begin{cases}
        \dfrac{x^4 - x^2 y^2}{(x^2 + y^2)^2}, & (x, y) \neq (0, 0)\\
        0, & (x, y) = (0, 0)
    \end{cases}
\]
\[
    \pdm{f}{x}{y} = 
    \begin{cases}
        \dfrac{2x^3 y^2 - 2x y^4}{(x^2 + y^2)^3}, & (x, y) \neq (0, 0)\\
        0, & (x, y) = (0, 0)
    \end{cases}
\]
\[
    \pdm{f}{y}{x} = 
    \begin{cases}
        \dfrac{2x^3 y^2 - 2x y^4}{(x^2 + y^2)^3}, & (x, y) \neq (0, 0)\\
        DNE, & (x, y) = (0, 0)
    \end{cases}
\]
\[
    \pdm{f}{x}{x} = 
    \begin{cases}
        \dfrac{2y^3(x^2 - y^2)}{(x^2 + y^2)^3}, & (x, y) \neq (0, 0)\\
        0, & (x, y) = (0, 0)
    \end{cases}
\]
\[
    \pdm{f}{y}{y} = 
    \begin{cases}
        \dfrac{x^2(2x^2 - y^2)}{(x^2 + y^2)^3}, & (x, y) \neq (0, 0)\\
        0, & (x, y) = (0, 0)
    \end{cases}
\]

\section{Chain Rule}

Given two functions \(f(x, y)\) and \(f(a, b)\) where $a$ and $b$ are dependent on $x$ and $y$, the chain rule states:

The partial derivative of the composite function with respect to the first new variable is equal to the partial derivative of the composite function with respect to the first old variable multiplied by partial derivative of the same old variable with respect to the current new variable plus.

\subsection{Examples:}

\begin{enumerate}
    
    \item Find $\pd{z}{r}$:
        \begin{align*}
        z(r,\theta) &= f(x, y)\\
        \text{where } x &= r\cos\theta,\\
        \text{ and } y &= r\sin\theta\\
        \pd{z}{r} &= \pd{f}{x}\pd{x}{r} + \pd{f}{y}\pd{y}{r}\\
        &= \pd{f}{x}\cos\theta + \pd{f}{y}\sin\theta
    \end{align*}
    We do not have $\pd{f}{x}$ or $\pd{f}{y}$, so we cannot find the exact value of $\pd{z}{r}$.

    \item $g(u, v, \ldots) = f(x, y, \ldots)$
    
    \begin{align*}
        &\pd{g}{u} = \pd{f}{x}\pd{x}{u} + \pd{f}{y}\pd{y}{u} + \ldots\\
        &\pd{g}{v} = \pd{f}{x}\pd{x}{v} + \pd{f}{y}\pd{y}{v} + \ldots\\
        &\,\,\,\vdots
    \end{align*}

    \item $z = f(x, y)$, $x = s + t$, $y = s - t$
    
        
        \[
            \text{Show that: }\quad \left(\pd{z}{x}\right)^2 - \left(\pd{z}{y}\right)^2 = \pd{z}{s} \cdot \pd{z}{t}
        \]
        \[
            z(s, t) = z(x, y)
        \]
        \begin{align*}
            \pd{z}{s} &= \pd{z}{x}\cdot\pd{x}{s} + \pd{z}{y}\cdot\pd{y}{s}\\
            &= \pd{z}{x} + \pd{z}{y}\\\\
            \pd{z}{t} &= \pd{z}{x}\cdot\pd{x}{t} + \pd{z}{y}\cdot\pd{y}{t}\\
            &= \pd{z}{x} - \pd{z}{y}\\\\
            \pd{z}{s}\cdot\pd{z}{t} &= \left(\pd{z}{x} + \pd{z}{y}\right)\left(\pd{z}{x} - \pd{z}{y}\right)\\
            &= \left(\pd{z}{x}\right)^2 - \left(\pd{z}{y}\right)^2\\
            \therefore \pd{z}{s}\cdot\pd{z}{t} &= \left(\pd{z}{x}\right)^2 - \left(\pd{z}{y}\right)^2
        \end{align*}

\end{enumerate}

\section{Second Derivative Test of Multiple Variables}

The second derivative test is used to determine the nature of a critical point of a function of multiple variables. The test is as follows:

\begin{enumerate}
    \item Compute the second partial derivatives of the function \(f(x, y)\): 
    \[
    f_{xx}, \ f_{yy}, \ \text{and} \ f_{xy}.
    \]

    \item Form the \textbf{Hessian matrix}:
    \[
    H = \begin{bmatrix}
    f_{xx} & f_{xy} \\
    f_{xy} & f_{yy}
    \end{bmatrix}.
    \]

    \item Calculate the \textbf{determinant of the Hessian matrix} (\(\Delta f\)):
    \[
    \delta f = f_{xx}f_{yy} - (f_{xy})^2.
    \]

    \item Use the following rules to classify the critical point:
    \begin{itemize}
        \item If \(\Delta f > 0\) and \(f_{xx} > 0\), the critical point is a \textbf{local minimum}.
        \item If \(\Delta f > 0\) and \(f_{xx} < 0\), the critical point is a \textbf{local maximum}.
        \item If \(\Delta f < 0\), the critical point is a \textbf{saddle point}.
        \item If \(\Delta f = 0\), the test is \textbf{inconclusive}.
    \end{itemize}
\end{enumerate}




\subsection{Examples}
1. 
 
\[
    f(x, y) = 4x + 6y - x^2 - y^2
\]
\[ 
    D = \{ (x, y), \, x \in [0, 4], \, y \in [0, 5]\}
\]
\[
df = 4dx + 6dy - 2xdx - 2ydy
\]
\[
    \grad{f} = \langle 4-2x, 6-2y \rangle
\]
\[
    \grad{f} = \langle 0, 0 \rangle \implies x = 2, y = 3
\]
\[  
    \pdn{2}{f}{x} = -2, \quad \pdn{2}{f}{y} = -2, \quad \pdm{f}{x}{y} = 0
\]
\[
    \Delta f(x, y) = -4 < 0 \implies \text{ local maximum at } (2, 3)
\]
\[
\text{Vertices: } O = (0, 0), \, A = (4, 0), \, B = (4, 5), \, C = (0, 5)
\]
OA: 
\[
    OA, \, x \in (0, 4) \implies f(x, 0) = 4x - x^2
\]
\[
    f'(x) = 4 - 2x = 0 \implies x = 2
\]
\[
    f(2, 0) = 8 \implies \text{ local maximum at } (2, 0)
\]
AB:
\[
    AB, \, y \in (0, 5) \implies f(4, y) = 6y - y^2
\]
\[
    f'(y) = 6 - 2y = 0 \implies y = 3
\]
\[
    f(4, 3) = 18 \implies \text{ local maximum at } (4, 3)
\]
BC:
\[
    BC, \, x \in (0, 4) \implies f(x, 5) = 4x - x^2
\]
\[
    f'(x) = 4 - 2x = 0 \implies x = 2
\]
\[
    f(2, 5) = 8 \implies \text{ local maximum at } (2, 5)
\]
CO:
\[
    CO, \, y \in (0, 5) \implies f(0, y) = 6y - y^2
\]
\[
    f'(y) = 6 - 2y = 0 \implies y = 3
\]
\[
    f(0, 3) = 18 \implies \text{ local maximum at } (0, 3)
\]
\begin{center}
    \begin{tabular}{|c|c|c|}
        \hline
        Boundary & Point & $f(x, y)$ \\
        \hline
        OA & (2, 0) & 4 \\
        OA & (0, 0) & 0 \\
        AB & (4, 3) & 9 \\
        AB & (4, 0) & 0 \\
        BC & (2, 5) & 9 \\
        BC & (4, 5) & 5 \\
        CO & (0, 3) & 9 \\
        CO & (0, 5) & 5 \\
           & (2, 3) & 13 \\
        \hline
    \end{tabular}
\end{center}
\[
    \therefore \text{ global maximum at } (2, 3, 13)
\]

\section{Lagrange Multipliers}

The method of \textbf{Lagrange multipliers} is used to find the maximum and minimum values of a function \( f(x, y, \dots) \) subject to one or more constraints. The process is as follows:

\begin{enumerate}
    \item Define the \textbf{Lagrangian function}:
    \[
    \mathcal{L}(x, y, \lambda) = f(x, y, \dots) + \lambda \cdot g(x, y, \dots),
    \]
    where \( g(x, y, \dots) = 0 \) is the constraint and \( \lambda \) is the \textbf{Lagrange multiplier}.

    \item Compute the partial derivatives of \(\mathcal{L}\) with respect to all variables and \(\lambda\):
    \[
    \frac{\partial \mathcal{L}}{\partial x}, \ \frac{\partial \mathcal{L}}{\partial y}, \ \dots, \ \frac{\partial \mathcal{L}}{\partial \lambda}.
    \]

    \item Set these derivatives equal to zero to form a system of equations:
    \[
    \begin{aligned}
    &\frac{\partial \mathcal{L}}{\partial x} = 0, \\
    &\frac{\partial \mathcal{L}}{\partial y} = 0, \\
    &\,\,\,\,\vdots \\
    &\frac{\partial \mathcal{L}}{\partial \lambda} = g(x, y, \dots) = 0.
    \end{aligned}
    \]

    \item Solve the system of equations to find the critical points.

    \item Evaluate \( f(x, y, \dots) \) at the critical points to determine the maximum or minimum values, depending on the problem.
\end{enumerate}

\textbf{Key Notes:}
\begin{itemize}
    \item The method can be extended to functions with multiple constraints. For \(k\) constraints \( g_1(x, y, \dots) = 0, \ g_2(x, y, \dots) = 0, \ \dots, \ g_k(x, y, \dots) = 0\), introduce a Lagrange multiplier \(\lambda_i\) for each constraint and define:
    \[
    \mathcal{L}(x, y, \lambda_1, \lambda_2, \dots) = f(x, y, \dots) + \lambda_1 \cdot g_1(x, y, \dots) + \lambda_2 \cdot g_2(x, y, \dots) + \dots.
    \]
    \item Always verify that the solutions satisfy the constraints.
\end{itemize}

\subsection{Examples}
1. Find the maximum and minimum values of \(V = xyz\) subject to the constraint \(x + y + z = 1\).

\begin{align*}
    \mathcal{L}(x, y, z, \lambda) &= xyz + \lambda(x + y + z - 1)\\
    \frac{\partial \mathcal{L}}{\partial x} &= yz + \lambda = 0,\\
    \frac{\partial \mathcal{L}}{\partial y} &= xz + \lambda = 0,\\
    \frac{\partial \mathcal{L}}{\partial z} &= xy + \lambda = 0,\\
    \frac{\partial \mathcal{L}}{\partial \lambda} &= x + y + z - 1 = 0.
\end{align*}

Solving the system of equations, we find \(x = y = z = \dfrac{1}{3}\). Thus, the maximum and minimum values of \(V\) are \(\dfrac{1}{27}\).

\subsection{Shortcut}
The Lagrange multiplier method can be simplified by using the following steps:
\begin{enumerate}
    \item Solve the constraint for each variable
    \subitem If it has multiple constraints, solve for the system of the constraints.
    \item Substitute the variables into the function
    \item Find the critical points
    \item Evaluate the function at the critical points
    \item Compare the values to find the maximum and minimum
\end{enumerate}

\subsubsection{Examples}
1. $f(x, y) = x^2 + y^2$, $xy = 1$
\[
    y = \dfrac{1}{x} \implies f(x) = x^2 + \dfrac{1}{x^2}
\]
\[
    f'(x) = 2x - \dfrac{2}{x^3} = 0 \implies x = 1
\]
\[
    f(1) = 2 \implies \text{ local minimum at } (1, 1)
\]
2. $f(x, y) = 3x + y$, $x^2 + y^2 = 10$
\[
    f(x, y) =
    \begin{cases}
        x = \sqrt{10}\cos\theta,\\
        y = \sqrt{10}\sin\theta
    \end{cases}
\]
\[
    f(\sqrt{10}\cos\theta, \sqrt{10}\sin\theta) = 3\sqrt{10}\cos\theta + \sqrt{10}\sin\theta
\]
\[
    f'(\theta) = -3\sqrt{10}\sin\theta + \sqrt{10}\cos\theta = 0 \implies \theta = \dfrac{5\pi}{6}
\]
\[
    f\left(\sqrt{10}\cos\left(\dfrac{5\pi}{6}\right), \sqrt{10}\sin\left(\dfrac{5\pi}{6}\right)\right) = -\sqrt{10} 
\]
\[
    \implies \text{ local minimum at } \left(\sqrt{10}\cos\left(\dfrac{5\pi}{6}\right), \sqrt{10}\sin\left(\dfrac{5\pi}{6}\right)\right)
\]